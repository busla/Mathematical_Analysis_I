\documentclass[icelandic,a4paper,12pt]{article}
\usepackage{beamerarticle}

\mode<presentation>
{
  \usetheme{boxes}
  % með efnisyfirliti: Szeged, Frankfurt 
  % án efnisyfirlits: Pittsburgh
  % áhugavert: CambridgeUS, Boadilla
  %\setbeamercovered{transparent} %gegnsætt
  \setbeamercovered{invisible}
  }

\usepackage[english,icelandic]{babel}
\usepackage[utf8]{inputenc}
\usepackage{t1enc}
\selectlanguage{icelandic}
\usepackage{graphicx}
\usepackage{amsmath}
\usepackage{amssymb}
\usepackage{mathrsfs}
% \newcommand{\C}{{\mathbb  C}}
% \newcommand{\Z}{{\mathbb Z}}
% \newcommand{\R}{{\mathbb  R}}
% \newcommand{\N}{{\mathbb  N}}
% \newcommand{\Q}{{\mathbb Q}}
\renewcommand{\phi}{\varphi}
\renewcommand{\epsilon}{\varepsilon}

%\usepackage{pgfpages}
% \pgfpagesuselayout{2 on 1}[a4paper,border shrink=5mm]

\def\lecturename{Stærðfræðigreining IB}
\title{\insertlecture}
\author{Benedikt Steinar Magnússon, \href{mailto:bsm@hi.is}{bsm@hi.is}}
\institute
{
  Verkfræði- og náttúruvísindasvið\\
  Háskóli Íslands
}
\subtitle{Stærðfræðigreining IB}
%\subject{\lecturename}

\mode<article>
{
	\usepackage[colorlinks=false,
	pdfauthor={Benedikt Steinar Magnusson},
	pdftitle={IB: Namsefni
	}]{hyperref}
  %\usepackage{times}
  %\usepackage{mathptmx}
  \usepackage[left=1.5cm,right=4cm,top=1.5cm,bottom=3cm]{geometry}
}

% Beamer version theme settings

%\useoutertheme[height=0pt,width=2cm,right]{sidebar}
%\usecolortheme{rose,sidebartab}
%\useinnertheme{circles}
%\usefonttheme[only large]{structurebold}

\setbeamercolor{sidebar right}{bg=black!15}
\setbeamercolor{structure}{fg=blue}
\setbeamercolor{author}{parent=structure}

\setbeamerfont{title}{series=\normalfont,size=\LARGE}
\setbeamerfont{title in sidebar}{series=\bfseries}
\setbeamerfont{author in sidebar}{series=\bfseries}
\setbeamerfont*{item}{series=}
\setbeamerfont{frametitle}{size=}
\setbeamerfont{block title}{size=\small}
\setbeamerfont{subtitle}{size=\normalsize,series=\normalfont}

\defbeamertemplate*{footline}{infolines theme}
 {
   \leavevmode%
   \hbox{%
   \begin{beamercolorbox}[wd=.333333\paperwidth,ht=2.25ex,dp=1ex,center]{author in head/foot}%
   %  \usebeamerfont{author in head/foot}\insertshortauthor~~\beamer@ifempty{\insertshortinstitute}{}{(\insertshortinstitute)}
   \end{beamercolorbox}%
   \begin{beamercolorbox}[wd=.333333\paperwidth,ht=2.25ex,dp=1ex,center]{title in head/foot}%
    % \usebeamerfont{title in head/foot}\insertshorttitle
   \end{beamercolorbox}%
   \begin{beamercolorbox}[wd=.333333\paperwidth,ht=2.25ex,dp=1ex,right]{date in head/foot}%
     %\usebeamerfont{date in head/foot}\insertshortdate{}\hspace*{2em}
     \insertshortlecture.\insertframenumber{} / \insertshortlecture.\inserttotalframenumber\hspace*{2ex} 
   \end{beamercolorbox}}%
   \vskip0pt%
 }
  


\setbeamertemplate{sidebar right}
{
  {\usebeamerfont{title in sidebar}%
    \vskip1.5em%
    \hskip3pt%
    \usebeamercolor[fg]{title in sidebar}%
    \insertshorttitle[width=2cm-6pt,center,respectlinebreaks]\par%
    \vskip1.25em%
  }%
  {%
    \hskip3pt%
    \usebeamercolor[fg]{author in sidebar}%
    \usebeamerfont{author in sidebar}%
    \insertshortauthor[width=2cm-2pt,center,respectlinebreaks]\par%
    \vskip1.25em%
  }%
  \hbox to2cm{\hss\insertlogo\hss}
  \vskip1.25em%
  \insertverticalnavigation{2cm}%
  \vfill
  \hbox to 2cm{\hfill\usebeamerfont{subsection in
      sidebar}\strut\usebeamercolor[fg]{subsection in
      sidebar}\insertshortlecture.\insertframenumber\hskip5pt}%
  \vskip3pt%
}%

\setbeamertemplate{title page}
{
  \vbox{}
  \vskip1em
  %{\huge Kapitel \insertshortlecture\par}
  {\usebeamercolor[fg]{title}\usebeamerfont{title}\inserttitle\par}%
  \ifx\insertsubtitle\@empty%
  \else%
    \vskip0.25em%
    {\usebeamerfont{subtitle}\usebeamercolor[fg]{subtitle}\insertsubtitle\par}%
  \fi%     
  \vskip1em\par
  %Vorlesung \emph{\lecturename}\ vom 
  \insertdate\par
  \vskip0pt plus1filll
  \leftskip=0pt plus1fill\insertauthor\par
  \insertinstitute\vskip1em
}

%\logo{\includegraphics[width=2cm]{beamerexample-lecture-logo.pdf}}



% Article version layout settings

\mode<article>

\makeatletter
\def\@listI{\leftmargin\leftmargini
  \parsep 0pt
  \topsep 5\p@   \@plus3\p@ \@minus5\p@
  \itemsep0pt}
\let\@listi=\@listI


\setbeamertemplate{frametitle}{\paragraph*{\insertframetitle\
    \ \small\insertframesubtitle}\ \par
}
\setbeamertemplate{frame end}{%
  \marginpar{\scriptsize\hbox to 1cm{\sffamily%
      \hfill\strut\insertshortlecture.\insertframenumber}\hrule height .2pt}}
\setlength{\marginparwidth}{1cm}
\setlength{\marginparsep}{1.5cm}

\def\@maketitle{\makechapter}

\def\makechapter{
  \newpage
  \null
  \vskip 2em%
  {%
    \parindent=0pt
    \raggedright
    \sffamily
    \vskip8pt
    %{\fontsize{36pt}{36pt}\selectfont Kapitel \insertshortlecture \par\vskip2pt}
    {\fontsize{24pt}{28pt}\selectfont \color{blue!50!black} \insertlecture\par\vskip4pt}
    {\Large\selectfont \color{blue!50!black} \insertsubtitle, \@date\par}
    \vskip10pt

    \normalsize\selectfont \@author\par\vskip1.5em
    %\hfill BLABLA
  }
  \par
  \vskip 1.5em%
}

\let\origstartsection=\@startsection
\def\@startsection#1#2#3#4#5#6{%
  \origstartsection{#1}{#2}{#3}{#4}{#5}{#6\normalfont\sffamily\color{blue!50!black}\selectfont}}

\makeatother

\mode
<all>




% Typesetting Listings

\usepackage{listings}
\lstset{language=Java}

\alt<presentation>
{\lstset{%
  basicstyle=\footnotesize\ttfamily,
  commentstyle=\slshape\color{green!50!black},
  keywordstyle=\bfseries\color{blue!50!black},
  identifierstyle=\color{blue},
  stringstyle=\color{orange},
  escapechar=\#,
  emphstyle=\color{red}}
}
{
  \lstset{%
    basicstyle=\ttfamily,
    keywordstyle=\bfseries,
    commentstyle=\itshape,
    escapechar=\#,
    emphstyle=\bfseries\color{red}
  }
}



% Common theorem-like environments

\theoremstyle{definition}
\newtheorem{exercise}[theorem]{\translate{Exercise}}




% New useful definitions:

\newbox\mytempbox
\newdimen\mytempdimen

\newcommand\includegraphicscopyright[3][]{%
  \leavevmode\vbox{\vskip3pt\raggedright\setbox\mytempbox=\hbox{\includegraphics[#1]{#2}}%
    \mytempdimen=\wd\mytempbox\box\mytempbox\par\vskip1pt%
    \fontsize{3}{3.5}\selectfont{\color{black!25}{\vbox{\hsize=\mytempdimen#3}}}\vskip3pt%
}}

\newenvironment{colortabular}[1]{\medskip\rowcolors[]{1}{blue!20}{blue!10}\tabular{#1}\rowcolor{blue!40}}{\endtabular\medskip}

\def\equad{\leavevmode\hbox{}\quad}

\newenvironment{greencolortabular}[1]
{\medskip\rowcolors[]{1}{green!50!black!20}{green!50!black!10}%
  \tabular{#1}\rowcolor{green!50!black!40}}%
{\endtabular\medskip}



\lecture[1]{2. Markgildi og samfelldni}{lecture-text}
\date{29. ágúst 2015}

\newcommand{\C}{{\mathbb  C}}
\newcommand{\Z}{{\mathbb Z}}
\newcommand{\R}{{\mathbb  R}}
\newcommand{\N}{{\mathbb  N}}
\newcommand{\Q}{{\mathbb Q}}
\newcommand{\Sin}{{\text{Sin}}}
\newcommand{\Tan}{{\text{Tan}}}
\newcommand{\Cos}{{\text{Cos}}}
\newcommand{\Cosh}{{\text{Cosh}}}
\newcommand{\arsinh}{{\text{arsinh}}}
\newcommand{\arcosh}{{\text{arcosh}}}
\newcommand{\artanh}{{\text{artanh}}}

\begin{document}
\setcounter{tocdepth}{2}
\tableofcontents

\section{Veldaraðir}

\subsection{Veldaraðir}
\subsubsection{Skilgreining}  
Röð á forminu 
$$\sum_{n=0}^\infty a_n(x-c)^n=a_0+a_1(x-c)+a_2(x-c)^2+\cdots$$
kallast \emph{veldaröð} með \emph{miðju} í punktinum $c$.

\subsubsection{Setning: Samleitnibil} 
Um sérhverja veldaröð $\sum_{n=0}^\infty a_n(x-c)^n$ gildir eitt af þrennu:
\begin{enumerate}[(i)]
\item Röðin er aðeins samleitin fyrir $x=c$.
\item Til er jákvæð tala $R$ þannig að veldaröðin er alsamleitin
fyrir öll $x$ þannig að $|x-c|<R$ og ósamleitin fyrir öll $x$
þannig að $|x-c|>R$.  Mögulegt að röðin sé samleitin í öðrum eða
báðum punktunum $x=c-R$ og $x=c+R$, en það er líka mögulegt að röðin
sé ósamleitin í þeim báðum.
\item  Röðin er samleitin fyrir allar rauntölur $x$.
\end{enumerate}

\subsubsection{Skilgreining: Miðja og samleitnigeisli}
Látum $\sum_{n=0}^\infty a_n(x-c)^n$ vera veldaröð.
\begin{enumerate}[(i)]
\item[(a)]  Talan $c$ kallast \emph{miðja} eða \emph{samleitnimiðja} veldaraðarinnar. 
\item[(b)]  Í tilviki (ii) í Setningu 25.2 er röðin alsamleitin á bilinu $(c-R, c+R)$ 
og ósamleitin fyrir utan bilið $[c-R, c+R]$.
  
Mögulegt er að röðin sé samleitin (alsamleitin eða skilyrt samleitin) í öðrum eða 
báðum punktunum $x=c-R$ og $x=c+R$ (þetta þarf að athuga sérstaklega).  

Talan $R$ er kölluð \emph{samleitnigeisli} raðarinnar.  

.. todo::
  Laga tilvísanir

Í tilfelli 25.2(i) þegar röðin er bara samleitin fyrir $x=c$ setjum við $R=0$  
og í tilfelli 25.2(iii) þegar röðin er samleitin fyrir allar rauntölur $x$ þá 
setjum við $R=\infty$.
\item[(c)]  \emph{Samleitnibil} veldaraðarinnar $\sum_{n=0}^\infty a_n(x-c)^n$ 
er mengi allra gilda $x$ þannig að röðin er samleitin. Setning 25.2 sýnir að 
þetta mengi er alltaf bil.
\end{enumerate}

\subsubsection{Athugasemd}
\begin{enumerate}[(i)]
\item Í tilfelli 25.2(i) er samleitnibilið $\{c\}$.
\item Í tilfelli 25.2(ii) eru fjórir möguleikar eftir því hvort röðin er
samleitin í hvorugum, öðrum eða báðum punktunum $x=c-R$ og $x=c+R$.  
Samleitnibilið getur verið 
\begin{enumerate}[(i)]
\item $(c-R, c+R)$,
\item $[c-R, c+R)$,
\item $(c-R, c+R]$,  
\item $[c-R, c+R]$.
\end{enumerate}
\item Í tilfelli 25.2(iii) er samleitnibilið $(-\infty, \infty)$.
\end{enumerate}

\subsection{Samleitnipróf}
\subsubsection{Setning}  
Látum $\sum_{n=0}^\infty a_n(x-c)^n$ vera veldaröð.
\begin{enumerate}[(i)]
\item[(i)]  \emph{Kvótapróf:}  Gerum ráð fyrir að 
$L=\lim_{n\rightarrow\infty}\left|\frac{a_{n+1}}{a_n}\right|$ sé til
eða $\infty$.  

Þá hefur veldaröðin  $\sum_{n=0}^\infty a_n(x-c)^n$ samleitnigeisla 
$$R= \left\{\begin{array}{ll}
\infty & \mbox{ef }L=0,\\
\frac{1}{L} & \mbox{ef }0<L<\infty,\\
0 & \mbox{ef }L=\infty.\\ 
\end{array} \right.
$$
\item[(ii)]  \emph{Rótarpróf:}  Gerum ráð fyrir að 
$L=\lim_{n\rightarrow\infty}\sqrt[n]{|a_n|}$ sé til
eða $\infty$.  
Þá hefur veldaröðin  $\sum_{n=0}^\infty a_n(x-c)^n$
samleitnigeisla 
$$R= \left\{\begin{array}{ll}
\infty & \mbox{ef }L=0,\\
\frac{1}{L} & \mbox{ef }0<L<\infty,\\
0 & \mbox{ef }L=\infty.\\
\end{array}
\right.
$$
\end{enumerate}

\subsubsection{Setning Abels}
Fallið $f$ skilgreint á samleitnibili með 
$$
f(x)=\sum_{n=0}^\infty a_n(x-c)^n
$$  
er samfellt á öllu samleitnibili veldaraðarinnar.  

Ef samleitnigeislinn er $0<R<\infty$ og röðin er samleitin í punktinum $x=c+R$ 
þá er 
$$\lim_{x\rightarrow (c+R)^-}f(x)=f(c+R)=\sum_{n=0}^\infty
a_n((c+R)-c)^n=\sum_{n=0}^\infty a_nR^n.$$

Eins ef röðin er samleitin í punktinum $x=c-R$ þá er
$$\lim_{x\rightarrow (c-R)^+}f(x)=f(c-R)=\sum_{n=0}^\infty
a_n((c-R)-c)^n=\sum_{n=0}^\infty a_n(-R)^n.$$

\subsubsection{26.1 Setning: Diffrað lið fyrir lið}
Látum $\sum_{n=0}^\infty a_n(x-c)^n=a_0+a_1(x-c)+a_2(x-c)^2+a_3(x-c)^3+\cdots$
vera veldaröð með miðju í $c$ og samleitnigeisla $R$. 

Fyrir $x\in(c-R, c+R)$ skilgreinum við 
$$
f(x)=\sum_{n=0}^\infty a_n(x-c)^n.
$$

Fallið $f$ er diffranlegt og 
$$f'(x)=\sum_{n=1}^\infty na_n(x-c)^{n-1}=a_1+2a_2(x-c)+3a_3(x-c)^2+\cdots$$
og röðin fyrir $f'(x)$ er samleitin fyrir öll $x\in(c-R, c+R)$.

Þetta þýðir að við getum diffrað veldaraðir lið fyrir lið.

\subsection{Samfelldni}
Þar sem diffranleg föll eru samfelld þá fæst eftirfarandi.

\subsubsection{Fylgisetning}
Fallið $f$ er samfellt á $(c-R, c+R)$.

\subsubsection{Setning: Heildað lið fyrir lið}
Látum $\sum_{n=0}^\infty a_n(x-c)^n=a_0+a_1(x-c)+a_2(x-c)^2+a_3(x-c)^3+\cdots$
vera veldaröð með miðju í $c$ og samleitnigeisla $R$. 

Fyrir $x\in(c-R, c+R)$ skilgreinum við $f(x)=\sum_{n=0}^\infty a_n(x-c)^n$.

Fallið $f$ hefur stofnfall 
\begin{multline*}
F(x)=\sum_{n=0}^\infty \frac{a_n}{n+1}(x-c)^{n+1} \\
=a_0(x-c)+\frac{a_1}{2}(x-c)^2+\frac{a_2}{3}(x-c)^3+
\frac{a_3}{4}(x-c)^4+\cdots
\end{multline*}
og röðin fyrir $F(x)$  er samleitin fyrir öll $x\in(c-R, c+R)$.

Þetta þýðir að við getum heildað veldaraðir lið fyrir lið.

\subsubsection{Setning}
Látum $\sum_{n=0}^\infty a_n(x-c)^n=a_0+a_1(x-c)+a_2(x-c)^2+a_3(x-c)^3+\cdots$
vera veldaröð með miðju í $c$ og samleitnigeisla $R$. 

Fyrir $x\in(c-R, c+R)$ skilgreinum við 
$$f(x)=\sum_{n=0}^\infty a_n(x-c)^n.$$

Fallið $f$ er $k$-sinnum diffranlegt fyrir $k=1, 2, 3, \ldots$ og
$$a_k=\frac{f^{(k)}(c)}{k!}.$$

\subsubsection{Skilgreining: Fágað fall}  
Fall $f$ þannig að til er veldaröð $\sum_{n=0}^\infty a_n(x-c)^n$ með
samleitnigeisla $R>0$ þannig að $$f(x)=\sum_{n=0}^\infty a_n(x-c)^n$$
fyrir öll $x\in(c-R, c+R)$ kallast \emph{fágað} (raunfágað) í punktinum $c$.

\subsubsection{Athugasemd}
Dæmi um raunfáguð föll eru margliður, ræð föll, hornaföll, veldisföll og lograr. 

\subsection{Taylorraðir}
\subsubsection{Skilgreining: Taylorröð}
Gerum ráð fyrir að fall $f(x)$ sé óendanlega oft diffranlegt í
punktinum $x=c$,  (þ.e.a.s.~$f^{(k)}(c)$ er til fyrir $k=0, 1, 2, \ldots$).  

Veldaröðin 
\begin{align*}
\sum_{n=0}^\infty \frac{f^{(n)}(c)}{n!}(x-c)^n = & f(c)+f'(c)(x-c)+
\frac{f''(c)}{2}(x-c)^2 \\ & + \frac{f'''(c)}{3!}(x-c)^3 
+ \frac{f^{(iv)}(c)}{4!}(x-c)^4 + \cdots 
\end{align*}
kallast \emph{Taylorröð} með miðju í $x=c$ fyrir $f(x)$.  

Ef svo vill til að $c=0$ þá er oft talað um \emph{Maclaurinröð}.

\subsubsection{Setning}  
Taylormargliða með miðju í $c$ fyrir $f$ er skilgreind sem margliðan
\begin{align*}
  P_n(x)& =\sum_{n=0}^n \frac{f^{(k)}(c)}{n!}(x-c)^n \\
  &=f(c)+f'(c)(x-c)+ \frac{f''(c)}{2}(x-c)^2+\cdots+\frac{f^{(n)}(c)}{n!}(x-c)^n.
\end{align*}

Skekkjan í $n$-ta stigs Taylornálgun er $R_n(x)=f(x)-P_n(x)$.  

Til er tala $X$ sem liggur á milli $c$ og $x$ þannig að 
$$
R_n(x)=\frac{f^{(n+1)}(X)}{(n+1)!}(x-c)^{n+1}.
$$

\subsubsection{Setning}  
Gerum ráð fyrir að $f$ sé fall sem er óendanlega oft diffranlegt í punktinum $c$. 

Fyrir fast gildi á $x$ þá er Taylorröðin 
$$
\sum_{n=0}^\infty \frac{f^{(n)}(c)}{n!}(x-c)^n
$$ 
samleitin með summu $f(x)$  ef og aðeins ef 
$$
\lim_{n\rightarrow\infty}R_n(x)=0.
$$

\subsubsection{Dæmi: Tvíliðuröðin}
Fyrir $x$ þannig að $|x|<1$ og rauntölu $r$ gildir að 
\begin{align*}
(1+x)^r =& 1+rx+\frac{r(r-1)}{2!}x^2+ \frac{r(r-1)(r-2)}{3!}x^3 \\ 
&+\frac{r(r-1)(r-2)(r-3)}{4!}x^4+\cdots\\
=& 1+ \sum_{n=1}^\infty \frac{r(r-1)(r-2)\cdots(r-n+1)}{n!}x^n.
\end{align*}

\subsubsection{Athugasemd}
Ef $r \in \N$ þá gefur summan að ofan einfaldlega stuðlanna þegar búið er að 
margfalda upp úr svigum, og summan er því endanleg því þegar $n \geq r+1$ þá
verða stuðlarnir 0.

Ef hins vegar $r\notin \N$ þá er enginn stuðlanna 0.

\subsubsection{Taylorraðir nokkra falla}
\begin{align*}
e^x&=\sum_{n=0}^\infty\frac{x^n}{n!}
    =1+x+\frac{x^2}{2}+\frac{x^3}{3!}
    +\cdots
  &\mbox{fyrir öll }x\\
\sin x&=  \sum_{n=0}^\infty\frac{(-1)^n}{(2n+1)!}x^{2n+1}
    =x-\frac{x^3}{3!}+\frac{x^5}{5!}-\frac{x^7}{7!}+\cdots
    &\mbox{fyrir öll }x\\ 
\cos x&=  \sum_{n=0}^\infty\frac{(-1)^n}{(2n)!}x^{2n}
    =1-\frac{x^2}{2!}+\frac{x^4}{4!}-\frac{x^6}{6!}+\cdots
    &\mbox{fyrir öll }x\\
\frac{1}{1-x}&=\sum_{n=0}^\infty x^n
    =1+x+x^2+x^3+\cdots
&\mbox{fyrir }-1<x<1\\
\frac{1}{(1-x)^2}&=\sum_{n=1}^\infty nx^{n-1}
    =1+2x+3x^2+4x^3+\cdots
&\mbox{fyrir }-1<x<1\\
\ln(1+x)&=  \sum_{n=1}^\infty\frac{(-1)^{n-1}}{n}x^n
    =x-\frac{x^2}{2}+\frac{x^3}{3}-\frac{x^4}{4}+\cdots
    &\mbox{fyrir }-1<x\leq 1\\
\tan^{-1} x&=  \sum_{n=0}^\infty\frac{(-1)^n}{2n+1}x^{2n+1}
    =x-\frac{x^3}{3}+\frac{x^5}{5}-\frac{x^7}{7}+\cdots
    &\mbox{fyrir }-1\leq x\leq 1\\\\
\sinh x&=  \sum_{n=0}^\infty\frac{x^{2n+1}}{(2n+1)!}
    =x+\frac{x^3}{3!}+\frac{x^5}{5!}+\frac{x^7}{7!}+\cdots
    &\mbox{fyrir öll } x\\
\cosh x&=  \sum_{n=0}^\infty\frac{x^{2n}}{(2n)!}
    =1+\frac{x^2}{2!}+\frac{x^4}{4!}+\frac{x^6}{6!}+\cdots
    &\mbox{fyrir öll } x\\
\end{align*} 

\end{document}
