\mode<presentation>
{
  \usetheme{boxes}
  % með efnisyfirliti: Szeged, Frankfurt 
  % án efnisyfirlits: Pittsburgh
  % áhugavert: CambridgeUS, Boadilla
  %\setbeamercovered{transparent} %gegnsætt
  \setbeamercovered{invisible}
  }

\usepackage[english,icelandic]{babel}
\usepackage[utf8]{inputenc}
\usepackage{t1enc}
\selectlanguage{icelandic}
\usepackage{graphicx}
\usepackage{amsmath}
\usepackage{amssymb}
\usepackage{mathrsfs}
% \newcommand{\C}{{\mathbb  C}}
% \newcommand{\Z}{{\mathbb Z}}
% \newcommand{\R}{{\mathbb  R}}
% \newcommand{\N}{{\mathbb  N}}
% \newcommand{\Q}{{\mathbb Q}}
\renewcommand{\phi}{\varphi}
\renewcommand{\epsilon}{\varepsilon}

%\usepackage{pgfpages}
% \pgfpagesuselayout{2 on 1}[a4paper,border shrink=5mm]

\def\lecturename{Stærðfræðigreining IB}
\title{\insertlecture}
\author{Benedikt Steinar Magnússon, \href{mailto:bsm@hi.is}{bsm@hi.is}}
\institute
{
  Verkfræði- og náttúruvísindasvið\\
  Háskóli Íslands
}
\subtitle{Stærðfræðigreining IB}
%\subject{\lecturename}

\mode<article>
{
	\usepackage[colorlinks=false,
	pdfauthor={Benedikt Steinar Magnusson},
	pdftitle={IB: Namsefni
	}]{hyperref}
  %\usepackage{times}
  %\usepackage{mathptmx}
  \usepackage[left=1.5cm,right=4cm,top=1.5cm,bottom=3cm]{geometry}
}

% Beamer version theme settings

%\useoutertheme[height=0pt,width=2cm,right]{sidebar}
%\usecolortheme{rose,sidebartab}
%\useinnertheme{circles}
%\usefonttheme[only large]{structurebold}

\setbeamercolor{sidebar right}{bg=black!15}
\setbeamercolor{structure}{fg=blue}
\setbeamercolor{author}{parent=structure}

\setbeamerfont{title}{series=\normalfont,size=\LARGE}
\setbeamerfont{title in sidebar}{series=\bfseries}
\setbeamerfont{author in sidebar}{series=\bfseries}
\setbeamerfont*{item}{series=}
\setbeamerfont{frametitle}{size=}
\setbeamerfont{block title}{size=\small}
\setbeamerfont{subtitle}{size=\normalsize,series=\normalfont}

\defbeamertemplate*{footline}{infolines theme}
 {
   \leavevmode%
   \hbox{%
   \begin{beamercolorbox}[wd=.333333\paperwidth,ht=2.25ex,dp=1ex,center]{author in head/foot}%
   %  \usebeamerfont{author in head/foot}\insertshortauthor~~\beamer@ifempty{\insertshortinstitute}{}{(\insertshortinstitute)}
   \end{beamercolorbox}%
   \begin{beamercolorbox}[wd=.333333\paperwidth,ht=2.25ex,dp=1ex,center]{title in head/foot}%
    % \usebeamerfont{title in head/foot}\insertshorttitle
   \end{beamercolorbox}%
   \begin{beamercolorbox}[wd=.333333\paperwidth,ht=2.25ex,dp=1ex,right]{date in head/foot}%
     %\usebeamerfont{date in head/foot}\insertshortdate{}\hspace*{2em}
     \insertshortlecture.\insertframenumber{} / \insertshortlecture.\inserttotalframenumber\hspace*{2ex} 
   \end{beamercolorbox}}%
   \vskip0pt%
 }
  


\setbeamertemplate{sidebar right}
{
  {\usebeamerfont{title in sidebar}%
    \vskip1.5em%
    \hskip3pt%
    \usebeamercolor[fg]{title in sidebar}%
    \insertshorttitle[width=2cm-6pt,center,respectlinebreaks]\par%
    \vskip1.25em%
  }%
  {%
    \hskip3pt%
    \usebeamercolor[fg]{author in sidebar}%
    \usebeamerfont{author in sidebar}%
    \insertshortauthor[width=2cm-2pt,center,respectlinebreaks]\par%
    \vskip1.25em%
  }%
  \hbox to2cm{\hss\insertlogo\hss}
  \vskip1.25em%
  \insertverticalnavigation{2cm}%
  \vfill
  \hbox to 2cm{\hfill\usebeamerfont{subsection in
      sidebar}\strut\usebeamercolor[fg]{subsection in
      sidebar}\insertshortlecture.\insertframenumber\hskip5pt}%
  \vskip3pt%
}%

\setbeamertemplate{title page}
{
  \vbox{}
  \vskip1em
  %{\huge Kapitel \insertshortlecture\par}
  {\usebeamercolor[fg]{title}\usebeamerfont{title}\inserttitle\par}%
  \ifx\insertsubtitle\@empty%
  \else%
    \vskip0.25em%
    {\usebeamerfont{subtitle}\usebeamercolor[fg]{subtitle}\insertsubtitle\par}%
  \fi%     
  \vskip1em\par
  %Vorlesung \emph{\lecturename}\ vom 
  \insertdate\par
  \vskip0pt plus1filll
  \leftskip=0pt plus1fill\insertauthor\par
  \insertinstitute\vskip1em
}

%\logo{\includegraphics[width=2cm]{beamerexample-lecture-logo.pdf}}



% Article version layout settings

\mode<article>

\makeatletter
\def\@listI{\leftmargin\leftmargini
  \parsep 0pt
  \topsep 5\p@   \@plus3\p@ \@minus5\p@
  \itemsep0pt}
\let\@listi=\@listI


\setbeamertemplate{frametitle}{\paragraph*{\insertframetitle\
    \ \small\insertframesubtitle}\ \par
}
\setbeamertemplate{frame end}{%
  \marginpar{\scriptsize\hbox to 1cm{\sffamily%
      \hfill\strut\insertshortlecture.\insertframenumber}\hrule height .2pt}}
\setlength{\marginparwidth}{1cm}
\setlength{\marginparsep}{1.5cm}

\def\@maketitle{\makechapter}

\def\makechapter{
  \newpage
  \null
  \vskip 2em%
  {%
    \parindent=0pt
    \raggedright
    \sffamily
    \vskip8pt
    %{\fontsize{36pt}{36pt}\selectfont Kapitel \insertshortlecture \par\vskip2pt}
    {\fontsize{24pt}{28pt}\selectfont \color{blue!50!black} \insertlecture\par\vskip4pt}
    {\Large\selectfont \color{blue!50!black} \insertsubtitle, \@date\par}
    \vskip10pt

    \normalsize\selectfont \@author\par\vskip1.5em
    %\hfill BLABLA
  }
  \par
  \vskip 1.5em%
}

\let\origstartsection=\@startsection
\def\@startsection#1#2#3#4#5#6{%
  \origstartsection{#1}{#2}{#3}{#4}{#5}{#6\normalfont\sffamily\color{blue!50!black}\selectfont}}

\makeatother

\mode
<all>




% Typesetting Listings

\usepackage{listings}
\lstset{language=Java}

\alt<presentation>
{\lstset{%
  basicstyle=\footnotesize\ttfamily,
  commentstyle=\slshape\color{green!50!black},
  keywordstyle=\bfseries\color{blue!50!black},
  identifierstyle=\color{blue},
  stringstyle=\color{orange},
  escapechar=\#,
  emphstyle=\color{red}}
}
{
  \lstset{%
    basicstyle=\ttfamily,
    keywordstyle=\bfseries,
    commentstyle=\itshape,
    escapechar=\#,
    emphstyle=\bfseries\color{red}
  }
}



% Common theorem-like environments

\theoremstyle{definition}
\newtheorem{exercise}[theorem]{\translate{Exercise}}




% New useful definitions:

\newbox\mytempbox
\newdimen\mytempdimen

\newcommand\includegraphicscopyright[3][]{%
  \leavevmode\vbox{\vskip3pt\raggedright\setbox\mytempbox=\hbox{\includegraphics[#1]{#2}}%
    \mytempdimen=\wd\mytempbox\box\mytempbox\par\vskip1pt%
    \fontsize{3}{3.5}\selectfont{\color{black!25}{\vbox{\hsize=\mytempdimen#3}}}\vskip3pt%
}}

\newenvironment{colortabular}[1]{\medskip\rowcolors[]{1}{blue!20}{blue!10}\tabular{#1}\rowcolor{blue!40}}{\endtabular\medskip}

\def\equad{\leavevmode\hbox{}\quad}

\newenvironment{greencolortabular}[1]
{\medskip\rowcolors[]{1}{green!50!black!20}{green!50!black!10}%
  \tabular{#1}\rowcolor{green!50!black!40}}%
{\endtabular\medskip}



