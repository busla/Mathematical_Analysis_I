\documentclass[12pt]{article}

\usepackage[T1]{fontenc} \usepackage[icelandic]{babel}
\usepackage{latexsym,amssymb,amsmath}

\voffset=-1.0in
\hoffset=-0.5in
\textwidth=6in
\textheight=10.0in

\begin{document}

\pagestyle{empty}

\newcommand{\dive}{\mbox{${\rm div}$}}
\newcommand{\rot}{\mbox{${\rm rot}$}}
\newcommand{\R}{\mbox{${\bf R}$}}
\newcommand{\Ov}{\mbox{${\bf 0}$}}
\newcommand{\rv}{\mbox{${\bf r}$}}
\newcommand{\vv}{\mbox{${\bf v}$}}
\newcommand{\av}{\mbox{${\bf a}$}}
\newcommand{\bv}{\mbox{${\bf b}$}}
\newcommand{\cv}{\mbox{${\bf c}$}}
\newcommand{\iv}{\mbox{${\bf i}$}}
\newcommand{\jv}{\mbox{${\bf j}$}}
\newcommand{\kv}{\mbox{${\bf k}$}}
\newcommand{\Rn}{\mbox{${\bf R}^n$}}
\newcommand{\Rm}{\mbox{${\bf R}^m$}}

\centerline{\sc \LARGE Stærðfræðigreining IB}
\medskip
\centerline{\sc \large Að læra stærðfræði.}
\bigskip
\medskip


{\bf Að lesa:}    
Í fyrirlestrum gefst aðeins 
tími til að fara yfir helstu atriði námsefnisins og verðið þið að
að kynna ykkur stóran hluta þess upp á eigin spýtur. Sumir nemendur
hafa farið í gegnum framhaldsskóla með því
læra utan að reikniaðferðir og vart reynt að skilja námsefnið.  Hættan
við þessa námsaðferð er að allt fari
í einn graut, og 
nemendur geti ekki yfirfært þekkingu sína á önnur svipuð verkefni.
Því held ég að léttasta leiðin í gegnum stærðfræðinámskeiðin í námi
ykkar sé að skilja efnið.  Skilningur á efninu fæst með því að rýna í
skilgreiningar og reglur, skoða sannanir og tengja við dæmi.  
Þið {\bf verðið} að lesa
kennslubókina og kynna ykkur efni fyrirlestra.  
Stór hluti þess sem þið munuð fást við í
háskólanámi ykkar er aðeins skiljanlegur þegar notað er tungumál
stærðfræðinnar.  Ef þið leggið það á ykkur að verða læs á tungumál
stærðfræðinnar þá munið þið njóta þess í öllu ykkar námi.

{\bf Að reikna:}  
Dæmaskammtarnir eru stórir.  Mörg dæmanna eru hugsuð 
sem léttar reikniæfingar.
Önnur dæmi eru til að æfa  
meðferð hugtaka og að hjálpa ykkur að skilja 
skilgreiningarnar.  Það er ekki nóg að læra niðurstöður, reglur og 
reikniaðferðir: til að geta beitt þeim af öryggi þarf að hafa góðan 
skilning á þeim grundvallarhugtökunum.
 
Til að hafa fullt gagn af dæmatímunum þurfið þið að reyna við dæmin
áður en þið mætið í dæmatímann.
Ég hvet ykkur eindregið til að vinna saman í náminu.  Þannig getur
maður fengið hjálp þegar maður er strand og
einnig skerpir fátt skilning manns  jafn mikið og að útskýra
fyrir öðrum.  Námið verður  skemmtilegra og þannig
léttara.  

{\bf Einbeiting:}  Meiri árangur næst í náminu ef þið eruð einbeitt.
Það er hægt að blekkja sjálfan sig í að halda að maður hafi verið að
læra allan daginn þegar í raun var deginum eitt í spjall við félagana,
netvafr, fésbókar stúss, msn, tölvuleiki, hlusta á ipodinn, og
svo framvegis.   


{\bf Frágangur skiladæma:}  
Leggið áherslu á vandaða og agaða framsetningu á lausnum
skiladæmanna.  Það að setja lausnina skýrt og skipulega fram er
nauðsynlegt til að maður sjálfur skilji lausnina til hlítar.  
   
Líkt og venjulegt tal- og ritmál þá hefur mál stærðfræðinnar sína 
málfræði, t.d.\ krefst táknið ,,$=$\rq\rq\ þess að sitthvoru megin 
við það standi stærðir eða stærðtákn, og ef fullyrðing sem er sett fram 
er rétt þá eru þessar stærðir jafnar.  Sitthvoru megin við 
táknið ,,$\Rightarrow$\rq\rq\ 
varða að standa fullyrðingar, og þegar það 
er notað rétt þá er fullyrðing hægra megin afleiðing  
fullyrðingarinnar  vinstra megin, þ.e.a.s.\ alltaf þegar fullyrðing 
vinstra megin er sönn þá er fullyrðingin hægrra megin líka sönn.

Táknin  ,,$\Rightarrow$\rq\rq, ,,$\Leftrightarrow$\rq\rq\ 
eru hentug þegar 
útreikningar eru sýndir á töflu, en mín ráðlegging er að nota þau sem 
minnst.   
Þau eru ekki notuð í kennslubókinni, ekki heldur í 
lausnaheftinu, og atvinnustærðfræðingar nota 
þessi tákn ekki í sínum skrifum.  Í  löngum
útreikngum er oft hægt að nota  ,,$=$\rq\rq\  í stað leiðingaörva. 
Engin ástæða er heldur til að nota 
táknin ,,$\vee$\rq\rq, ,,$\wedge$\rq\rq\ því 
orðin ,,eða\rq\rq\ og {,,}og\rq\rq\ eru 
mun skýrari;  það eina sem táknin hafa fram yfir orðin er tilgerðin.

%\vfill\eject

Gott er að hafa eftirfarandi reglur í huga þegar gengið er frá lausnum 
verkefna:

1. Textinn á að vera ein samfelld heild sem fullnægir sömu kröfum og 
gerðar eru til annars ritaðs máls.  Stærðfræðiformúla eða stærðtákn 
á aldrei að koma fyrir eitt sér, heldur  alltaf 
að vera felt inn í samfellt mál.

2. Uppsetningin á að vera aðlaðandi og frágangur snyrtilegur.

3. Allar fullyrðingar skulu studdar ljósum rökum.

4. Svara þarf því sem spurt er um!  Það þarf að koma
   skýrt fram hvert svarið er. 


\vfill
\hfill  Rögnvaldur G. Möller
\end{document}
